\documentclass[12pt]{article}

\usepackage{graphicx}
\usepackage[english]{babel}
\usepackage[utf8]{inputenc}
\usepackage{hyperref}

\title{CMPE 300 Homework 2}
\author{Ibrahim Ozgurcan OZTAS}
\date{20/11/2019}

\begin{document}
\maketitle
\newpage

\section{Question 1:}
  In Question 1,

\section{Question 2:}
  In Question 2, the recurrence relation for the worst case input n is; \\
  $ W(n) = W(\sqrt{n}) + 2\sqrt{n} $ \\

  The RootStack function has one recursive call and a while loop that iterates from $n-\sqrt{n}$ to 0. The beginning of the function body includes one comparison as $n=2$ in if part. The amount of iteration of the while loop is $\sqrt{n} - 1$ and it has 2 comparison operations per iteration. Hence, $2 \cdot (\sqrt{n}-1)$ comparisons are done in $\sqrt{n}-1$ iterations, therefore the current value of b is 0. In the next iteration, there is one more comparison which is the evaluation of b and 0, results in which the comparison returns false, and breaks the while loop done by the processor. Finally, the total comparison done in the function is $2\cdot(\sqrt{n}-1)+1+1 = 2\cdot\sqrt{n}$. \\\\
  The recursive call inside the function will be called with the list that has input size $\sqrt{n}$. Thus, the recurrence relation can be written as; \\\\
  $ W(n) = W(\sqrt{n}) + 2\sqrt{n} $ \\\\


\section{Question 3:}
  In Question 3, the answers are below. \\
  a) For this question, the characteristic equation of the recurrence relation can be written as $ \alpha^2=a\alpha+b. $ Therefore, the current characteristic equation is $ \alpha^2=4\alpha-3. $ The roots are $ \alpha_{1}=3 $ and $ \alpha_{2}=1 $. Hence, the function is $T(n)=c_{1}3^n+c_{2}$, since $1^n=1$ for all integer $ n \in Z^+.$ Then, checking the initial conditions sets the constants $c_{1}$ and $c_{2}$ to 4 and 5 respectively.

  \begin{itemize}
    \item[--] $ e_{1} \gets c_{1} + c_{2} = 9$ \\
          $ e_{2} \gets 3c_{1} + c_{2} = 17$ \\
          Substituting $e_{2}$ from $3e_{1}$ results in \newline $3e_{1} - e_{2} = 3c_{1} + 3c_{2} - 3c_{1} - c_{2} = 27 - 17 \Longrightarrow 2c_{2} = 10$, \newline which implies $\Longrightarrow c_{2} = 5$.
  \end{itemize} \\

  In conclusion, the function can be written in closed form as \newline $T(n) = 4 \cdot 3^n + 5$ \\

  b) For this question, the closed form of the recurrence relation can be written as $T(n) = [\sum_{i=0}^{n-1}(\frac{1}{2})^{i}] + 10 \cdot \frac{1}{2^{n}} $ \\

  \begin{itemize}
    \item[--] By backward substitution method, \\\\
              $T(n) = T(n-1) \cdot \frac{1}{2} + 1$ \\
              $T(n-1) = T(n-2) \cdot \frac{1}{2} + 1$ \\
              $T(n-2) = T(n-3) \cdot \frac{1}{2} + 1$ \\
              It continues until $T(0)$ is reached. (After n-1 iterations) \\
              $T(1) = T(0) \cdot \frac{1}{2} + 1$ \\\\
              Swapping every equation from $T(1)$ to $T(n-1)$ with the corresponding equation will result in; \\
              $T(n) = T(0) \cdot (\frac{1}{2})^n + 1 + \frac{1}{2} + \frac{1}{4} + \frac{1}{8} + ... + \frac{1}{2^{(n-1)}}$ \\\\
              Plug $T(0)$ into the equation; \\
              $T(n) = 10 \cdot (\frac{1}{2})^n + 1 + \frac{1}{2} + \frac{1}{4} + \frac{1}{8} + ... + \frac{1}{2^{(n-1)}}$ \\\\
              In the right side of the equation, there is the series of $x=\frac{1}{2}$ from power $0$ to $(n-1)$. Hence, it can be written as \\
              $T(0) = 10 \cdot (\frac{1}{2})^n + \sum_{i=1}^{n-1}(\frac{1}{2})^i$ (Check the geometric series from the literature.)\\\\
              The summation will result in $\frac{(\frac{1}{2})^n - 1}{\frac{1}{2} - 1} $ \\
              Hence, the function $T(n) = 2 + 8 \cdot (\frac{1}{2})^n$


  \end{itemize}

  c) For this question, the closed form of the recurrence relation can be written as $T(n) = \prod_{i=0}^{k-1}(\frac{n}{2^{i}})$, where $k=\log_{2}{n}$, since the assumption allows one to say $n=2^{k}$. \\\\
  Hence, $T(n) = \frac{n^{k}}{2^{\frac{k(k-1)}{2}}}$, where $k = \log_{2}n$. \\\\
  It can be refined into $T(n) = (\frac{n}{2^{\frac{(k-1)}{2}}})^{k}$, where $k = \log_{2}n$. \\\\
  The improved form of the function is $T(n) = (\frac{n}{2^{\frac{(\log_{2}n-1)}{2}}})^{\log_{2}n}$ \\\\
  The final form of the function is $T(n) = n^{\frac{1+\log_{2}n}{2}}$

  \begin{itemize}
    \item[--] By backward substition method, \\\\
              $T(n) = n \cdot T(n/2)$ \\
              $T(n/2) = \frac{n}{2} \cdot T(n/4)$ \\
              $T(n/4) = \frac{n}{4} \cdot T(n/8)$ \\
              It continues until $T(1)$ is reached. (After $k \gets \log_{2}n$ iterations.) \\
              $T(2) = 2 \cdot T(1)$ \\

              Swapping every equation from $T(1)$ to $T(n-1)$ with the corresponding equation will result in; \\
              $T(n) = T(1) \cdot n \cdot \frac{n}{2} \cdot \frac{n}{4} \cdot ... \cdot 2$ \\

              Plug $T(1)$ into the equation; \\

              $T(n) = 1 \cdot n \cdot \frac{n}{2} \cdot \frac{n}{4} \cdot ... \cdot 2$ \\

              The equation can be refined as; \\
              $T(n) = \prod_{i=0}^{k-1}(\frac{n}{2^{i}})$ \\
              Hence, $T(n) = \frac{n^{k}}{2^{(\sum_{i=0}^{k-1}i)}}$ \\\\

              $T(n) = \frac{n^{k}}{2^{\frac{k(k-1)}{2}}}$, where $k = \log_{2}n$.

              Improved form of the refined function is; \\
              $T(n) = (\frac{n}{2^{\frac{(k-1)}{2}}})^{k}$, where $k=\log_{2}n$ \\
              Replacing k in the previous function results in; \\\\
              $T(n) = (\frac{n}{2^{\frac{(\log_{2}n-1)}{2}}})^{\log_{2}n}$, which implies to $T(n) = n^{\frac{1+\log_{2}n}{2}}$

  \end{itemize}




\end{document}
